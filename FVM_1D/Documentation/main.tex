\documentclass{report}
\usepackage{geometry}
\geometry{margin=1.05in}
\usepackage{amsmath}
\usepackage[makeroom]{cancel}
%\numberwithin{equation}{section}
\usepackage{graphicx}
\usepackage{float}
\usepackage{subcaption}
\usepackage{empheq}
%\usepackage{slashbox}
\newcommand*\widefbox[1]{\fbox{\hspace{2em}#1\hspace{2em}}}
\usepackage{natbib}
\bibliographystyle{abbrvnat}
\usepackage{hyperref}
\usepackage{siunitx}
\usepackage[table,pdftex,dvipsnames]{xcolor}
\usepackage{animate}

\usepackage{multicol}
\hypersetup{
	colorlinks,
	linkcolor={red!50!black},
	citecolor={black},
	urlcolor={blue!80!black}
}
\graphicspath{{img/}}
\DeclareMathOperator\sech{sech}
\graphicspath{{img/}}
\usepackage{listings}
\usepackage{color}
\usepackage{ifthen}

\definecolor{dkgreen}{rgb}{0,0.6,0}
\definecolor{gray}{rgb}{0.5,0.5,0.5}
\definecolor{mauve}{rgb}{0.58,0,0.82}

%\lstset{frame=tb,
%	language=Matlab,
%	aboveskip=3mm,
%	belowskip=3mm,
%	showstringspaces=false,
%	columns=flexible,
%	basicstyle={\small\ttfamily},
%	numbers=none,
%	numberstyle=\tiny\color{gray},
%	keywordstyle=\color{blue},
%	commentstyle=\color{dkgreen},
%	stringstyle=\color{mauve},
%	breaklines=true,
%	breakatwhitespace=true,
%	tabsize=3
%}
%

\lstset{language=[5.3]lua,
	%	backgroundcolor=\color{backcolour},   
	commentstyle=\color{OliveGreen},
	keywordstyle=\color{RoyalBlue},
	numberstyle=\tiny\color{Gray},
	stringstyle=\color{Orange},
	basicstyle=\footnotesize\ttfamily,
	breaklines=true, breakatwhitespace=true,     
	breaklines=true,                 
	captionpos=t,                    
	keepspaces=true,                 
	numbers=left,                    
	numbersep=5pt,                  
	showspaces=false,                
	showstringspaces=false,
	showtabs=false,                  
	tabsize=2,
	frame=tb,
	prebreak=\raisebox{0ex}[0ex][0ex]{\ensuremath{\hookleftarrow}},
	postbreak=\raisebox{0ex}[0ex][0ex]{\ensuremath{\space\hookrightarrow\space}},
}




\setlength{\textwidth}{6.5in}
\setlength{\textheight}{8.5in}
\setlength{\evensidemargin}{0in}
\setlength{\oddsidemargin}{0in}
\setlength{\topmargin}{0in}

\setlength{\parindent}{0pt}
\setlength{\parskip}{0.1in}

\usepackage[section]{placeins}

\let\Oldsubsection\subsection
\renewcommand{\subsection}{\FloatBarrier\Oldsubsection}

\let\Oldsubsubsection\subsubsection
\renewcommand{\subsubsection}{\FloatBarrier\Oldsubsubsection}


%%%%%%%%%%%%%%%% Notes and things to add:
% Maybe use widetilde instead of tilde
% Clean up this header section over here
% Double check all equation labels and references.


% Do I even want a title page on these?
\title{1D Euler Equations using the Finite Volume Method}
\author{Chirag R. Skolar, PhD}
\date{\today}

\begin{document}
\maketitle
	
Some sort of intro here about Euler eqs.   
% Unsure how much detail I want to go into this. Do I start from first principles and derive it or what?
	
\begin{align}
	\frac{\partial \rho}{\partial t} + \nabla \cdot \big(\rho \mathbf{u} \big) &= 0  \label{eq:euler_cont} \\
	\frac{\partial \big( \rho \mathbf{u} \big) }{\partial t} + \nabla \cdot \big( \rho \mathbf{u} \mathbf{u} + p \mathbf{I} \big) &= 0  \label{eq:euler_mtm}  \\
	\frac{ \partial \mathcal{E}}{\partial t} + \nabla \cdot \big( [ \mathcal{E} + p] \mathbf{u} \big) &= 0 \label{eq:euler_energy} ,
\end{align}
where
\begin{equation}
	\mathcal{E} = \frac{p}{\gamma-1} + \frac{1}{2} \rho \mathbf{u} \cdot \mathbf{u}
\end{equation}
Eqs.~\ref{eq:euler_cont}-\ref{eq:euler_energy} can be rewritten as
\begin{equation}
	\frac{\partial \mathbf{Q}}{\partial t} + \nabla \cdot \mathbf{F} = 0 .\label{eq:hyperbolic} 
\end{equation}
The conservative variables are
\begin{equation}
	\mathbf{Q} = 
	\begin{bmatrix}
		Q_1 \\ Q_2 \\ Q_3 \\ Q_4 \\ Q_5
	\end{bmatrix} = 
	\begin{bmatrix}
		\rho \\
		\rho u \\
		\rho v \\
		\rho w \\
		\mathcal{E} 
	\end{bmatrix} = 
	\begin{bmatrix}
		V_1 \\
		V_1 V_2 \\
		V_1 V_3 \\
		V_1 V_4 \\
		\frac{V_5}{\gamma-1} + \frac{1}{2} V_1 \big( V_2^2 + V_3^2 + V_4^2 \big)
	\end{bmatrix}. \label{eq:cons_var}
\end{equation}
The primitive variables are   % Not sure if we need this but leave it in here for now
\begin{equation}
	\mathbf{V} = 
	\begin{bmatrix}
		V_1 \\ V_2 \\ V_3 \\ V_4 \\ V_5
	\end{bmatrix} = 
	\begin{bmatrix}
		\rho \\ u \\ v \\ w \\ p
	\end{bmatrix} = 
	\begin{bmatrix}
		Q_1 \\
		\frac{Q_2}{Q_1} \\
		\frac{Q_3}{Q_1} \\
		\frac{Q_4}{Q_1} \\
	(\gamma-1) \big[Q_5 - \frac{1}{2 Q_1} ( Q_2^2 + Q_3^2 + Q_4^2) \big]
	\end{bmatrix}. \label{eq:prim_var} 
\end{equation}
The flux tensor, $\mathbf{F}$ can be written as a combination of three column vectors,
\begin{equation}
	\mathbf{F}_x = 
	\begin{bmatrix}
		\rho u \\
		\rho u^2 + p \\
		\rho u v \\
		\rho u w \\
		( \mathcal{E} + p ) u 
	\end{bmatrix} =
	\begin{bmatrix}
		Q_2 \\
		\frac{Q_2^2}{Q_1} + (\gamma-1) \big[ Q_5 - \frac{1}{2 Q_1} (Q_2^2 + Q_3^2 + Q_4^2) \big] \\
		\frac{Q_2 Q_3}{Q_1} \\
		\frac{Q_2 Q_4}{Q_1} \\
		\frac{Q_2}{Q_1} \big[ \gamma Q_5 + \frac{1-\gamma}{2 Q_1} (Q_2^2+Q_3^2+Q_4^2) \big] 
	\end{bmatrix}
\end{equation}
\begin{equation}
	\mathbf{F}_y = 
	\begin{bmatrix}
		\rho v \\
		\rho uv \\
		\rho v^2 + p \\
		\rho v w \\
		( \mathcal{E} + p ) v
	\end{bmatrix} =
	\begin{bmatrix}
		Q_3 \\
		\frac{Q_2 Q_3}{Q_1} \\
		\frac{Q_3^2}{Q_1} + (\gamma-1) \big[ Q_5 - \frac{1}{2 Q_1} (Q_2^2 + Q_3^2 + Q_4^2) \big] \\
		\frac{Q_3 Q_4}{Q_1} \\
		\frac{Q_3}{Q_1} \big[ \gamma Q_5 + \frac{1-\gamma}{2 Q_1} (Q_2^2+Q_3^2+Q_4^2) \big] 
	\end{bmatrix}
\end{equation}
\begin{equation}
	\mathbf{F}_z = 
	\begin{bmatrix}
		\rho w \\
		\rho uw \\
		\rho v w \\
		\rho w^2 + p \\
		( \mathcal{E} + p ) w
	\end{bmatrix}
	\begin{bmatrix}
		Q_4 \\
		\frac{Q_2 Q_4}{Q_1} \\
		\frac{Q_3 Q_4}{Q_1} \\
		\frac{Q_4^2}{Q_1} + (\gamma-1) \big[ Q_5 - \frac{1}{2 Q_1} (Q_2^2 + Q_3^2 + Q_4^2) \big] \\
		\frac{Q_4}{Q_1} \big[ \gamma Q_5 + \frac{1-\gamma}{2 Q_1} (Q_2^2+Q_3^2+Q_4^2) \big] 
	\end{bmatrix}
\end{equation}

% Get flux jacobians. Maybe use maxima for this?

Get the characteristic speeds. Convert the equations into quasilinear form. 
Use the chain rule for the divergence term in Eq.~\ref{eq:hyperbolic} 
to get the derivatives onto the conservative variables.
\begin{equation}
	\frac{\partial \mathbf{Q}}{\partial t} + \frac{\partial \mathbf{F} }{\partial \mathbf{Q}} \nabla \cdot \mathbf{Q} = 0 \label{eq:quasilinear}
\end{equation}
The flux Jacobians are
\begin{equation}
	\frac{\partial \mathbf{F}_x}{\partial \mathbf{Q}} = 
	\begin{bmatrix}
		0 & 1 & 0 & 0 & 0\\  %%%%%%%%%%%%%%%%%%%%%%%%%%%%%%%%%%%%%%%%%%%%%%%%%%%%%%%%%%%%%%%%%%%%%%%%%%%%%%%%%%%%%%%%%%%%%%%%
		- \frac{Q_2^2}{Q_1^2} + \frac{\gamma-1}{2 Q_1^2} \big( Q_2^2 + Q_3^2 + Q_4^2 \big) & 
		(3-\gamma) \frac{Q_2}{Q_1} &
		(1-\gamma) \frac{Q_3}{Q_1} &
		(1-\gamma) \frac{Q_4}{Q_1} & 
		\gamma - 1  \\  %%%%%%%%%%%%%%%%%%%%%%%%%%%%%%%%%%%%%%%%%%%%%%%%%%%%%%%%%%%%%%%%%%%%%%%%
		- \frac{Q_2 Q_3 }{Q_1^2} & 
		\frac{Q_3}{Q_1} &
		\frac{Q_2}{Q_1} &
		0 & 0 \\ %%%%%%%%%%%%%%%%%%%%%%%%%%%%%%%%%%%%%%%%%%%%%%%%%%%%%%%%%%%%%%%%%%%%%%%%		 
		- \frac{Q_2 Q_4}{Q_1^2} & 
		\frac{Q_4}{Q_1} & 
		0 & 
		\frac{Q_2}{Q_1} & 0 \\ %%%%%%%%%%%%%%%%%%%%%%%%%%%%%%%%%%%%%%%%%%%%%%%%%%%%%%%%%%%%%%%%%%%%%%%%
		- \frac{\gamma  Q_2 Q_5}{Q_1^2} + (\gamma-1) \frac{Q_2}{Q_1^3} \big( Q_2^2 + Q_3^2 + Q_4^2 \big) &
		\gamma \frac{Q_5}{Q_1} + \frac{1-\gamma}{2 Q_1^2} \big( 3 Q_2^2 + Q_3^2 + Q_4^2 \big) &
		(1-\gamma) \frac{Q_2 Q_3}{Q_1^2}  & 
		(1-\gamma) \frac{Q_2 Q_4}{Q_1^2} & 
		\gamma \frac{Q_2}{Q_1}
	\end{bmatrix}
\end{equation}

The flux Jacobians in primitive variables are
\begin{equation}
	\frac{\partial \mathbf{F}_x}{\partial \mathbf{Q}} = 
	\begin{bmatrix}
		0  & 1 & 0 & 0 & 0   \\ %%%%%%%%%%%%%%%%%%%%%%%%%%%%%%%%%%%%%%%%%%%%%%%%%%%%%%%%%%%%%%%%%%%%%%%%
		\frac{\gamma-3}{2} u^2 + \frac{\gamma-1}{2} (v^2 + w^2) & 
		(3-\gamma) u & 
		(1-\gamma) v & 
		(1-\gamma) w & 
		\gamma - 1  \\ %%%%%%%%%%%%%%%%%%%%%%%%%%%%%%%%%%%%%%%%%%%%%%%%%%%%%%%%%%%%%%%%%%%%%%%%
		-uv & 
		v &
		u & 0 & 0  \\ %%%%%%%%%%%%%%%%%%%%%%%%%%%%%%%%%%%%%%%%%%%%%%%%%%%%%%%%%%%%%%%%%%%%%%%%
		- uw & 
		w & 0 & u & 0  \\ %%%%%%%%%%%%%%%%%%%%%%%%%%%%%%%%%%%%%%%%%%%%%%%%%%%%%%%%%%%%%%%%%%%%%%%%
		\frac{\gamma}{1-\gamma} \frac{ p u }{\rho} + \frac{\gamma -2}{2} u \big(u^2 + v^2 + w^2 \big) &
		\frac{\gamma}{\gamma-1} \frac{p}{\rho} + \frac{3-2\gamma}{2}u^2 + \frac{1}{2} \big(v^2+w^2) &
		(1-\gamma) uv &
		(1-\gamma) uw &
		\gamma u
	\end{bmatrix}
\end{equation}


Convert Eqs.~\ref{eq:euler_cont}-\ref{eq:euler_energy} to primitive form. First, begin with the continuity equation, Eq.~\ref{eq:euler_cont} and use the
product rule for the divergence term, which is $\nabla \cdot( \psi \mathbf{A} ) = \psi \nabla \cdot \mathbf{A}+ \mathbf{A}\cdot \nabla \psi$.
\begin{equation}
	\frac{\partial \rho}{\partial t} + \nabla \cdot \big( \rho \mathbf{u} \big) 
	= \frac{\partial \rho}{\partial t} + \rho \nabla \cdot \mathbf{u}+ \mathbf{u} \cdot \nabla \rho = 0
\end{equation}
Next do the momentum equation, Eq.~\ref{eq:euler_mtm} using the product rule on the temporal derivative term
and using $\nabla \cdot (\mathbf{u}\mathbf{v}) = (\nabla \cdot \mathbf{u})\mathbf{v} + \mathbf{u} \cdot \nabla \mathbf{v}$ on the divergence term.
Note that at one point, we use the continuity equation (Eq.~\ref{eq:euler_cont}) in this derivation.
\begin{align}
	\frac{\partial \rho \mathbf{u}}{\partial t} + \nabla \cdot \big(\rho \mathbf{u} \mathbf{u} + p \mathbf{I} \big) &=
	\rho \frac{\partial \mathbf{u}}{\partial t} + \mathbf{u}\frac{\partial \rho }{\partial t} + \nabla p + \nabla \cdot \big(\rho \mathbf{u} \mathbf{ u}\big) \\
	&= \mathbf{u} \Bigg[ \underbrace{\frac{\partial \rho}{\partial t} + \nabla \cdot \big(\rho \mathbf{u}\big) }_{0}\Bigg] + \rho \frac{\partial \mathbf{u}}{\partial t} + \rho \mathbf{u} \cdot \nabla \mathbf{u} + \nabla p = 0 \\
	&= \frac{\partial \mathbf{u}}{\partial t} + \mathbf{u} \cdot \nabla \mathbf{u}+ \frac{1}{\rho} \nabla p = 0
\end{align}
The energy equation, Eq.~\ref{eq:euler_energy}, requires a lot more work. First expand the energy term.
\begin{align}
	\frac{\partial \mathcal{E}}{\partial t} + \nabla \cdot \big( [ \mathcal{E} + p] \mathbf{u} \big)  &= 0 \\
	\frac{\partial }{\partial t} \Bigg[ \frac{p}{\gamma-1} + \frac{1}{2} \rho \mathbf{u} \cdot \mathbf{u} \Bigg]
	+ \nabla \cdot \Bigg[ \bigg( \frac{\gamma}{\gamma-1} + \frac{1}{2} \rho \mathbf{u}\cdot \mathbf{u} \bigg) \mathbf{u} \Bigg] &= 0
\end{align}
Next, split the equation into the pressure and kinetic energy parts and multiply the whole equation by $\gamma-1$
\begin{equation}
	 \frac{\partial p}{\partial t} + \gamma \nabla \cdot \big(p \mathbf{u} \big) 
	+ \frac{\gamma-1}{2} \Bigg[ \underbrace{\frac{\partial}{\partial t} \big(\rho \mathbf{u} \cdot \mathbf{u} \big) + \nabla \cdot \big( \mathbf{u} \rho \mathbf{u}\cdot \mathbf{u} \big)}_{R} \Bigg] = 0
\end{equation}
Now, we will just focus on $R$. Use the product rule and rearrange terms. Note that we substitute in the continuity equation in here as well.
\begin{equation}
	R = \mathbf{u}\cdot \mathbf{u} \Bigg[ \underbrace{ \frac{\partial \rho}{\partial t} + \rho \nabla \cdot \mathbf{u}+ \mathbf{u} \cdot \nabla \rho}_{0} \Bigg]
+ \rho \frac{\partial (\mathbf{u} \cdot \mathbf{u} )}{\partial t} + \rho \mathbf{u}\cdot \nabla \big(\mathbf{u}\cdot \mathbf{u} \big)
\end{equation}
Next, we use the product rule for the temporal derivative term and the identity for the gradient of a dot product, $\nabla \mathbf{A}\cdot \mathbf{B} = \mathbf{A}\cdot \nabla \mathbf{B}+ \mathbf{B}\cdot \nabla \mathbf{A}$.
\begin{equation}
	R = 2 \rho \mathbf{u} \cdot \Bigg[ \frac{\partial \mathbf{u}}{\partial t} + \mathbf{u} \cdot \nabla \mathbf{u} \Bigg]
\end{equation}
Inside the brackets, add and subtract $\rho^{-1} \nabla p$ to get the momentum equation.
\begin{align}
	R &= 2 \rho \mathbf{u} \cdot \Bigg[ \underbrace{\frac{\partial \mathbf{u}}{\partial t} + \mathbf{u} \cdot \nabla \mathbf{u} + \frac{1}{\rho} \nabla p }_{0}- \frac{1}{\rho }\nabla p \Bigg] \\
	&= - 2 \mathbf{u}\cdot \nabla p
\end{align}
Now, we can plug $R$ back in and use the product rule for divergence.
\begin{align}
	\frac{\partial p}{\partial t} + \gamma \nabla \cdot \big(p \mathbf{u} \big) 
	+ \frac{\gamma-1}{2} R &= 
	\frac{\partial p}{\partial t} + \gamma p \nabla \cdot \mathbf{u} + \gamma \mathbf{u}\cdot \nabla p 
	- \frac{\gamma-1}{\cancel{2}} \cancel{2} \mathbf{u}\cdot \nabla p \\
	&= \frac{\partial p}{\partial t} + \gamma p \nabla \cdot \mathbf{u} + \mathbf{u}\cdot \nabla p = 0
\end{align}
Thus, the Euler equations in primitive form are
\begin{align}
	\frac{\partial \rho}{\partial t} + \rho \nabla \cdot \mathbf{u}+ \mathbf{u} \cdot \nabla \rho &= 0\\
	\frac{\partial \mathbf{u}}{\partial t} + \mathbf{u} \cdot \nabla \mathbf{u}+ \frac{1}{\rho} \nabla p &= 0 \\
	\frac{\partial p}{\partial t} + \gamma p \nabla \cdot \mathbf{u} + \mathbf{u}\cdot \nabla p &= 0
\end{align}
These equations are already in the quasilinear form from Eq.\ref{eq:quasilinear}. 
\begin{equation} % See if we can rewrite this in a better way
	\frac{\partial \mathbf{V}}{\partial t} + \tilde{\mathbf{A}}_x \frac{\partial \mathbf{V}}{\partial x} + \tilde{\mathbf{A}}_y \frac{\partial \mathbf{V}}{\partial y} + \tilde{\mathbf{A}}_z \frac{\partial \mathbf{V}}{\partial z}
\end{equation}
Thus, we can write $\tilde{\mathbf{A}}$, which is the flux Jacobian for the primitive variables, in terms of its $x$, $y$, and $z$ components.
\begin{align}
	\tilde{\mathbf{A}}_x &= \begin{bmatrix}
		u & \rho & 0 & 0 & 0 \\
		0 & u & 0 & 0 & 1/\rho \\
		0 & 0 & u & 0 & 0 \\
		0 & 0 & 0 & u & 0 \\
		0 & \gamma p & 0 & 0 & u
	\end{bmatrix} \\
	\tilde{\mathbf{A}}_y &= \begin{bmatrix}
		v & 0 & \rho & 0 & 0 \\
		0 & v & 0 & 0 & 0 \\
		0 & 0 & v & 0 & 1/\rho \\
		0 & 0 & 0 & v & 0 \\
		0 & 0 & \gamma p & 0 & v
	\end{bmatrix} \\
	\tilde{\mathbf{A}}_z &= \begin{bmatrix}
		w & 0 & 0 & \rho & 0 \\
		0 & w & 0 & 0 & 0 \\
		0 & 0 & w & 0 & 0 \\
		0 & 0 & 0 & w & 1/\rho \\
		0 & 0 & 0 & \gamma p & w
	\end{bmatrix}
\end{align}
Next, find the eigenvalues of $\tilde{\mathbf{A}}_x$ to get the fastest speed in $x$.
We end up getting the following characteristic polynomial
\begin{equation}
	\big(u - \lambda\big)^3 \bigg[ \big( u - \lambda \big)^2 - \frac{\gamma p}{\rho} \big) \bigg] = 0
\end{equation}
Therefore, the 5 eigenvalues are
\begin{align}
	\lambda_{1,2,3} &= 0 \\
	\lambda_{4,5} &= u \pm \sqrt{\frac{\gamma p}{\rho}} = u \pm a
\end{align}
Note that $a=\sqrt{\frac{\gamma p}{\rho}}$ is the speed of sound.



Eventually, ICs from \cite{sod1978survey}.

% Maybe go through full determinant calculation? Can do this in the three directions. For now, just keep it with 1D.
% Review some linear algebra. There are definitely tricks that we can use to solve these determinants simpler?






\bibliographystyle{plainnat}
\bibliography{reference}
\addcontentsline{toc}{section}{Reference}
	
\end{document}
